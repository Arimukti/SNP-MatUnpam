\documentclass[12pt]{article}
\usepackage{amsmath, amssymb}
\usepackage{geometry}
\usepackage{booktabs}
\usepackage{graphicx}
\usepackage{caption}
\usepackage{array}

\usepackage{fancyhdr}
\setlength{\headheight}{14.5pt}
\addtolength{\topmargin}{-2.5pt}  % Opsional, agar tidak menambah total margin atas
\pagestyle{fancy}
\fancyhf{}
\fancyhead[L]{Statistik Non Parametrik}
\fancyhead[R]{Uji Chi Square dan Uji Median}
\fancyfoot[C]{\thepage}
\fancyfoot[R]{Universitas Pamulang}



\geometry{margin=1in}

\title{Modul Pembelajaran:\\Uji Chi-Kuadrat dan Uji Median untuk \textit{k} Sampel Independen}
\author{Prodi Matematika Unpam}
\date{\today}

\begin{document}
	
	\maketitle
	
	\section*{I. Tujuan Pembelajaran}
	Setelah mempelajari modul ini, mahasiswa diharapkan mampu:
	\begin{enumerate}
		\item Menjelaskan konsep dasar dan asumsi dari uji chi-kuadrat dan uji median.
		\item Mengidentifikasi situasi yang sesuai untuk menggunakan uji chi-kuadrat dan uji median pada \textit{k} sampel independen.
		\item Melakukan uji chi-kuadrat dan uji median untuk membandingkan \textit{k} sampel independen.
		\item Menafsirkan hasil pengujian secara statistik dan konteks.
	\end{enumerate}
	
	\section*{II. Deskripsi Singkat Materi}
	Dalam analisis data, kita sering ingin mengetahui apakah \textit{k} kelompok independen berasal dari populasi yang sama. Untuk data kategorik atau ordinal/tidak normal, digunakan pendekatan nonparametrik:
	
	\begin{itemize}
		\item \textbf{Uji Chi-Kuadrat}: untuk menguji perbedaan distribusi frekuensi antar kelompok (data kategorik).
		\item \textbf{Uji Median}: untuk menguji perbedaan lokasi tengah (median) antar \textit{k} kelompok independen, cocok untuk data ordinal/tak normal.
	\end{itemize}
	
	\section*{III. Uji Chi-Kuadrat untuk \textit{k} Sampel Independen}
	\subsection*{1. Tujuan}
	Menguji apakah ada perbedaan distribusi frekuensi antar \textit{k} kelompok.
	
	\subsection*{2. Asumsi}
	\begin{itemize}
		\item Data berupa frekuensi (bukan nilai kontinyu).
		\item Sampel independen.
		\item Frekuensi harapan tiap sel sebaiknya $\geq 5$.
	\end{itemize}
	
	\subsection*{3. Langkah-langkah}
	\begin{enumerate}
		\item Buat tabel kontingensi dari data.
		\item Hitung frekuensi harapan: $E_{ij} = \dfrac{\text{Total baris} \times \text{Total kolom}}{\text{Grand total}}$
		\item Hitung statistik:
		\[
		\chi^2 = \sum \frac{(O_{ij} - E_{ij})^2}{E_{ij}}
		\]
		\item Tentukan derajat kebebasan dan bandingkan dengan $\chi^2$ kritis.
		\item Tarik kesimpulan.
	\end{enumerate}
	
	\section*{IV. Uji Median untuk \textit{k} Sampel Independen}
	\subsection*{1. Tujuan}
	Menguji apakah median antar \textit{k} kelompok berbeda secara signifikan.
	
	\subsection*{2. Asumsi}
	\begin{itemize}
		\item Data ordinal atau numerik tak normal.
		\item Sampel independen.
	\end{itemize}
	
	\subsection*{3. Langkah-langkah}
	\begin{enumerate}
		\item Gabungkan semua data, tentukan median keseluruhan.
		\item Kategorikan tiap observasi: di atas atau di bawah median.
		\item Buat tabel kontingensi $2 \times k$.
		\item Hitung statistik $\chi^2$ dari tabel tersebut.
		\item Bandingkan dengan nilai kritis untuk mengambil keputusan.
	\end{enumerate}
	
	\section*{V. Contoh Soal dan Pembahasan}
	
	\subsection*{Contoh 1: Uji Chi-Kuadrat}
	\textbf{Soal:}  
	Sebuah survei preferensi minuman di 3 kota terhadap 3 jenis minuman menghasilkan tabel berikut:
	
	\begin{center}
		\begin{tabular}{lccc}
			\toprule
			& Teh & Kopi & Jus \\
			\midrule
			Kota A & 10 & 20 & 10 \\
			Kota B & 15 & 10 & 15 \\
			Kota C & 5 & 15 & 20 \\
			\bottomrule
		\end{tabular}
	\end{center}
	
	Apakah ada perbedaan preferensi antar kota?
	
	\textbf{Pembahasan:}  
	\begin{itemize}
		\item Hipotesis:
		\begin{itemize}
			\item $H_0$: Distribusi preferensi sama di semua kota.
			\item $H_1$: Ada perbedaan distribusi preferensi.
		\end{itemize}
		\item Hitung frekuensi harapan dan $\chi^2$:
		
		$$
		\chi^2 \approx 10, \quad df = (3-1)(3-1) = 4, \quad \chi^2_{0.05, 4} = 9.488
		$$
		\item Karena $10 > 9.488$, tolak $H_0$.
	\end{itemize}
	\textbf{Kesimpulan:} Terdapat perbedaan signifikan dalam preferensi minuman antar kota.
	
	\subsection*{Contoh 2: Uji Median}
	\textbf{Soal:}  
	Empat toko dinilai berdasarkan kepuasan pelanggan (1–10):
	
	\begin{itemize}
		\item Toko A: 6, 7, 5, 8
		\item Toko B: 9, 7, 6, 7
		\item Toko C: 4, 5, 4, 6
		\item Toko D: 7, 8, 9, 10
	\end{itemize}
	
	Ujilah apakah median kepuasan antar toko berbeda.
	
	\textbf{Pembahasan:}
	\begin{itemize}
		\item Gabungkan semua data: total 16 nilai, median keseluruhan = 7
		\item Kategorisasi: $\leq 7$ dan $>7$
		
		\begin{center}
			\begin{tabular}{lcc}
				\toprule
				Toko & $\leq 7$ & $>7$ \\
				\midrule
				A & 3 & 1 \\
				B & 3 & 1 \\
				C & 4 & 0 \\
				D & 1 & 3 \\
				\bottomrule
			\end{tabular}
		\end{center}
		
		\item Hitung $\chi^2 \approx 6.75$ dengan $df = 3$ dan $\chi^2_{0.05, 3} = 7.815$
		\item Karena $6.75 < 7.815$, gagal tolak $H_0$
	\end{itemize}
	
	\textbf{Kesimpulan:} Tidak terdapat perbedaan signifikan pada median kepuasan antar toko.
	
	\section*{VI. Latihan Soal}
	
	\textbf{Petunjuk:} Kerjakan soal-soal berikut untuk menguji pemahaman Anda terhadap materi uji Chi-Kuadrat dan uji Median untuk $k$ sampel independen.
	
	\begin{enumerate}
		\item Sebuah penelitian dilakukan untuk mengetahui apakah jenis metode pembelajaran (Tatap Muka, Daring, Hybrid) memengaruhi tingkat keberhasilan ujian mahasiswa (Lulus, Tidak Lulus). Hasil data sebagai berikut:
		
		\begin{center}
			\begin{tabular}{lcc}
				\toprule
				Metode & Lulus & Tidak Lulus \\
				\midrule
				Tatap Muka & 45 & 5 \\
				Daring & 30 & 10 \\
				Hybrid & 40 & 5 \\
				\bottomrule
			\end{tabular}
		\end{center}
		
		Gunakan uji Chi-Kuadrat untuk menentukan apakah metode pembelajaran berpengaruh terhadap kelulusan.
		
		\item Empat jenis pupuk digunakan pada lahan yang berbeda, dan hasil panen dalam kg sebagai berikut:
		
		\begin{itemize}
			\item Pupuk A: 25, 30, 28, 31
			\item Pupuk B: 35, 33, 32, 30
			\item Pupuk C: 20, 22, 21, 23
			\item Pupuk D: 29, 28, 31, 30
		\end{itemize}
		
		Gunakan uji median untuk menentukan apakah terdapat perbedaan median hasil panen antar pupuk.
		
		\item Suatu studi ingin mengetahui apakah tingkat pendidikan (SMA, D3, S1) berkaitan dengan jenis pekerjaan (PNS, Swasta, Wiraswasta). Data sebagai berikut:
		
		\begin{center}
			\begin{tabular}{lccc}
				\toprule
				Pendidikan & PNS & Swasta & Wiraswasta \\
				\midrule
				SMA & 10 & 20 & 15 \\
				D3 & 15 & 10 & 5 \\
				S1 & 25 & 15 & 10 \\
				\bottomrule
			\end{tabular}
		\end{center}
		
		Lakukan uji Chi-Kuadrat dan simpulkan apakah terdapat hubungan antara tingkat pendidikan dan jenis pekerjaan.
		
		\item Lima toko dinilai berdasarkan lama waktu pelayanan (dalam menit):
		
		\begin{itemize}
			\item Toko A: 5, 7, 6, 8
			\item Toko B: 9, 10, 8, 9
			\item Toko C: 6, 5, 7, 6
			\item Toko D: 10, 9, 11, 12
			\item Toko E: 7, 6, 8, 9
		\end{itemize}
		
		Gunakan uji median untuk mengetahui apakah waktu pelayanan rata-rata berbeda antar toko.
	\end{enumerate}
	
	
	\section*{VII. Referensi}
	\begin{itemize}
		\item Siegel, S., \& Castellan, N. J. (1988). \textit{Nonparametric Statistics for the Behavioral Sciences}.
		\item Conover, W. J. (1999). \textit{Practical Nonparametric Statistics}.
		\item Triola, M. F. (2018). \textit{Elementary Statistics}.
	\end{itemize}
	
\end{document}
